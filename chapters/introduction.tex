
\chapter{Introduction}

\section{Weathering of rocks}
Rocks formed deep inside the earth, when exposed on or near to the surface, adapt or re-equilibrate to their new surroundings by disintegration and decomposition through chemical and physical forces of the atmosphere, hydrosphere, and biosphere. This process is known as weathering  \citep{weathering}. The chemical weathering of rocks releases elements into the environment which may be absorbed by plants, deposited and sequestered in soils and other rocks, or end up in the oceans \citep{hellmann2012}. One family of minerals in particular, those belonging to Ca- and Mg- silicates\footnote{A group of silicon-oxygen compounds}, are known for their fast weathering rates and have influenced \ce{CO2} concentrations in the atmosphere over \SIrange[range-units = single,range-phrase = --]{e5}{e6}{y} timescales, thereby impacting the global carbon cycle \citep{Brantley2008b}. That, in turn, has influenced the global climate through feedback processes \citep{renforth2015}. Thus, quantifying silicate weathering rates and the associated global atmospheric \ce{CO2} concentration has been the focus of many past studies \citep{golubev2005}. The current interest in silicate dissolution rates stems from its application in \ce{CO2} removal techniques like enhanced weathering. 

The rates of silicate dissolution and factors controlling the rate have been studied through laboratory experiments for over half a century. However, the same methods when applied in the field have consistently produced much lower rates than observed in laboratory conditions \citep{white1995}. Factors explaining this divergent behaviour have not been successfully quantified \citep{renforth2015}. The question remains if laboratory results are good enough to produce desirable results for enhanced weathering application at field scale. The current thesis attempts to answer this question by undertaking long-term laboratory experiments closely mimicking the
proposed application.

\section{Structure of the thesis}
Chapter 2, Background and Literature Review, summarises the concepts and terminologies associated with rate of mineral dissolution and factors influencing it; followed by the motivation for the current work. Chapter 3, Materials and Methods, describes the laboratory setup and the instruments used to analyse physical and chemical properties of the mineral and water samples. Chapter 4, Results and Discussion, consists of results of the experiment, importantly the reaction rate, and discusses the variables affecting it. Chapter 5, Conclusions, summarises the findings in the light of past work and its implication to weathering research and enhanced weathering application . Additional information and calculations are presented in the Appendices.
